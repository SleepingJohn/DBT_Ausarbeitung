\documentclass[12pt]{article}\pagestyle{myheadings}

\title{Questionnaire}

\markright{Compiler Ausarbeitung}

\usepackage{amsmath,amssymb,amsthm,amsfonts,graphics}
\usepackage{listings}
\usepackage{graphics}
\graphicspath{../../../../../Pictures/}
\theoremstyle{plain}
\newtheorem{theorem}{Theorem}
\newtheorem{lemma}[theorem]{Lemma}
\newtheorem{corollary}[theorem]{Corollary}
\newtheorem{proposition}[theorem]{Proposition}
\newtheorem*{definition}{Definition}

\renewcommand{\qedsymbol}{\ensuremath{\blacksquare}}
\newcommand{\N}{\mathbb{N}}
\newcommand{\Z}{\mathbb{Z}}
\newcommand{\Q}{\mathbb{Q}}
\newcommand{\R}{\mathbb{R}}
\newcommand{\C}{\mathbb{C}} 

\begin{document}

\begin{enumerate}

\item \textbf{Discuss the 5 tuning-principals, give examples:} \\
\begin{itemize}
\item think globally, fix logically\\
global denken: das Problem finden und nicht nur dauerweise behandeln\\
lokal fixen mit minimalen Eingriffen um mögliceh sideeffects so gut es geht zu verhindern.\\
Zum Beispiel wenn Festplatte sehr überarbeitet ist was tun ?
\textbf{Lösungsweg a:} Man kauft einfach noch mehr Disks (local thinking)\\
Wenn man global denkt schaut man lieber wo ist die ganze Disk Aktivität generiert?\\
\item[-]fehlt ein index bei einem häufig auftretenden query? (dann fügt man einen query hinzu)
\item[-]der buffer der Datenbank ist zu klein ? so kann man den Buffer erhöhen.
\item[-]Protokoll und häufig genutzte Daten teilen sich die gleiche Festplatte? Einfach das Protokol auf eine andere Disk verschieben.\\
---$>$ Die Ursache zu beheben ist günsiter und effizienter als nur ständig die Symptome zu bekämpfen.\\
\textbf{Lösungsweg b:} Die Query mit der längsten Laufzeit bechleunigen.\\
Die langsamste Query ist vlt die unwichtigste und bringt kaum bessere Systemleistung! Besser ist es die wichtigen Querys zu beschleunigen.\\
\textbf{Lösungsweg c:} Die Query die gesamt die meiste Zeit benötigt beschleunigen.\\Die Query die die meiste Zeit braucht beschleunigen.
Vielleicht ist die Query unnütz und kann verworfen werden, bzw verbessert werden.\\ \\
\textit{Es ist wichtig auf das ganze System zu schauen (think globally)bei einer Fehlersuche und es dort zu beheben wo es auftritt(fix locally)}
\item partitioning breaks bottlenecks \\
Selten sind alle Teile eines Systems voll ausgeschöpft\\
oft schränkt ein teil die gesamte Perfomance ein dieses TEil ist das sogenannte Bottleneck.\\
Beispiel: Highway Traffic jam.
1)Fahrer schneller durch enge Abschnitte fahren lassen\\
2)Mehrere Bahnen mehrspurige Strecken.
3)Fahrer darauf hinweisen rush-hour zu meiden.
\\
-Mög 1 wär ein  lokaler fix.(add index)\\
-Mög 2 und 3 nennt man Partitionierung(Aufteilung).\\ \\
Zwei basic Partitionierungsstrategien sind:
\quad dividiere load über mehr Ressourcen(add lanes)\\
\quad verteile load über einen größeren Zeitraum (avoid rush-hour)\\ \\
Beispiel Problem: \\
-deadlocks zwingen längere Tranaktionen zum Abbruch
-lange Transaktionen verwenden alle Ressourcen(zb. Memory buffer)
Mögliche Lösung: In Zeit und verfügbaren Platz aufteilen.
\item[-]Zeitaufteilung: lange Transaktionen dann laufen lassen wenn wenig online Transaktionaktivität herrscht.\\
\item[-]Platzaufteilung: lange Transaktionen(wenn sie nur gelesen gehören) nur dann lesen on seperate hardware.\\
\item[-] serialize(reihenweise anordnen)lange Transaktionen so dass, diese nicht miteinader kolladieren.\\ \\
\includegraphics[scale=0.5]{\includegraphics[scale=1]{Partio.jpg} }



\item start-up costs are high runnig costs are low
\item render on the server what is due on the server
\item be prepared for trade -offs 
\end{itemize}


\item \textbf{} \\
\item \textbf{} \\
\item \textbf{} \\
\item \textbf{} \\
\item \textbf{} \\

\end{enumerate}

\end{document}
